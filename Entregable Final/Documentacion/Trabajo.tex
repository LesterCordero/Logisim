\documentclass[12pt,letterpaper]{article}
%\usepackage[spanish]{babel}
\usepackage[utf8]{inputenc}    %Uso de tildes si desarrolla en Linux
%\usepackage[latin1]{inputenc}   %Uso de tildes si desarrolla en Windows

\usepackage{setspace} %define comandos \singlespacing, \onehalfspacing, \doublespacing
\usepackage[left=1.5cm, right=1.5cm, top=1.5cm, bottom=1.5cm]{geometry}
\usepackage{graphicx}
\usepackage{tocstyle}
\usepackage{amssymb}
\usepackage{epsfig}
\usepackage{url}
\usepackage[pdftex,
	    breaklinks=true,
	    linktocpage=true,
	    pdfborder={0 0 0},
	    pdftoolbar=true,
	    colorlinks=true,
	    linkcolor=blue,
	    citecolor=blue,
	    filecolor=blue,
	    urlcolor=blue]{hyperref}

\begin{document}
\begin{tabular}{ccc}
\includegraphics[width=12mm]{logo.png}& 
\parbox{6in}{ \centering 
                  \textbf{Universidad de Costa Rica\\
                      Escuela de Ciencias de la Computación e Informática\\
                      CI-0119 Proyecto Integrador de Lenguaje Ensamblador y Arquitectura\\
                      II ciclo de 2018}\\
                     \hrulefill
                 }  & 
\includegraphics[width=15mm]{UCR.jpg}\\
\end{tabular}
\newline
\newline
\newline
\textbf{Integrantes del equipo:}\\
Lester Cordero Murillo - B62110\\
Josué Choso Rojas - B62045\\
Jesús Mena Amador - B54291\\
Emmanuel Palma Ortiz - B45147\\
\\
\large \textbf{Descripción}\\
Texto de prueba\\

\newpage

\tableofcontents
\section{Manual de Usuario}
\subsection{Introducción de esta sección}
\subsection{Utilización Básica}
\subsection{Explicación de la Pila}
\subsection{Operaciones Combinadas}
\subsection{Casos de Error}
\subsection{Rango Numérico Operable}
\subsection{Ejemplos de Uso}
\section{Manual de Desarrollador}
\subsection{Introducción de esta sección}
\subsection{Entrada y Salida de la Calculadora}
\subsection{Lista de Interrupciones}
\subsection{Conjunto de Instrucciones}
\subsection{Sistema de Memoria}
\subsection{Unidad Aritmética Lógica}
\subsection{Banderas de Respuesta}
\subsection{Circuitos Simples}
\subsection{Programa ensamblador de la Calculadora}

\end{document}
%------------------------------------------------------------
